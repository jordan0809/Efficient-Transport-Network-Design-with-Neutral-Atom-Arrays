\documentclass[10pt]{article}

\usepackage{amsmath}
\usepackage{amssymb}
\usepackage{tikz}
\usepackage{pgfplots}
\usepackage{cite}
\usepackage{fancyhdr}
%\usetikzlibrary{decorations.pathmorphing}
\usepackage{graphicx}
%\usepackage{braket}
%\usepackage{bbm}
%\usepackage[colorlinks=true]{hyperref}
%\usepackage{unicode-math}

\title{Title}
\author{Mudassir Moosa\\\date{ }}

\newcommand{\wpsi}{\widehat{\psi}}
\newcommand{\wwpsi}{\widetilde{\psi}}
\newcommand{\wrho}{\widehat{\rho}}
\newcommand{\wwrho}{\widetilde{\rho}}
\newcommand{\wS}{\widehat{S}}
\newcommand{\A}{\mathcal{A}}
\newcommand{\OO}{\mathcal{O}}
\newcommand{\p}{\hat{p}}
\newcommand{\x}{\hat{x}}
\newcommand{\im}{\int d^{d-2}y' \int_{0}^{\infty} dx' \,}
\newcommand{\imm}{\int d^{d-2}z' \int_{0}^{\infty} dx' \,}
\newcommand{\iz}{\int d^{d-2}z'}
\newcommand{\ix}{\int_{0}^{\infty} dx'}
\newcommand{\izz}{\int_{0}^{\infty} dz}
\newcommand{\nn}{\nonumber}
\newcommand{\tm}[2]{\begin{pmatrix} #1 \\ #2 \end{pmatrix}}
\newcommand{\bra}[1]{\langle #1|}
\newcommand{\ket}[1]{|#1\rangle}
\newcommand{\braket}[2]{\langle #1|#2\rangle}
\newcommand{\normsq}[1]{||#1||^{2}}
\newcommand{\ev}[1]{\langle #1 \rangle}

\oddsidemargin  0.0in
\evensidemargin 0.0in
\textwidth      6.5in
\headheight     -30pt%05pt
\topmargin      0.0in
\textheight=8.0in
\setlength{\parindent}{0in}

%\pagestyle{fancy}


\begin{document}

{\bf Calculation $\quad\quad\quad\quad\quad\quad\quad\quad\quad\quad\quad\quad\quad\quad\quad\quad\quad\quad\quad\quad\quad\quad\quad$ Mudassir Moosa}	\\

The best-known classical algorithm for solving the maximum weight independent set (MWIS) problem has a complexity of $O(1.2^{n})$ where $n$ is the number of nodes \cite{2013arXiv1312.6260X}. We take this complexity as an estimation of the number of flops needed to implement this algorithm. Now let us consider a graph with $n=250$ nodes. The estimate for the number of operations needed to find MWIS is then $1.2^{250} \, = \, 6.2\times 10^{19}$.\\

Now we consider a cluster of $3$ GPUs. Each GPU is known to operate at a rate of $2.6$ Tflops per second, i.e. $2.6 \times 10^{12}$ flops per second. So a cluster of $3$ GPUs can do $7.8\times 10^{12}$ flops per second. At this rate, $3$ GPUs can find the aforementioned MWIS in approximately $2000$ hours. \\

Moving on, let us consider an HPC that can operate at the rate of $7 $ Pflops per second, i.e., $7.0\times 10^{15}$. This HPC can solve the MWIS problem with $n=250$ in approximately $2.5$ hours.\\

Now let us consider a quantum computer (QC). We solve the MWIS problem by implementing the adiabatic evolution on an analog QC. Let us assume that we need $10^5$ shots of measurement to extract the desired result. During the Q/A session for the third round of this Hackathon, we learned that each shot of measurement can be done in $1$ second. Using the time to do $10^5$ measurements as a proxy of the total time taken, we deduce that it will take a QC of around $27$ hours to solve the MWIS problem. \\

We enter the estimates of the time that a GPU, HPC, and QC would take to solve the MWIS problem on the provided excel file. \\





\bibliographystyle{utcaps}
\bibliography{all}


\end{document}
